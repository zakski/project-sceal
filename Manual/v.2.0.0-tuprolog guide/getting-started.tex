%=======================================================================
\chapter{Getting Started}
\label{getting-started}
%=======================================================================

The \tuprolog{} distribution offers some tools either to consult and execute already existing Prolog programs, or to help developing new Prolog theories and interact with a Prolog engine. %
Depending on the use you would like to make of \tuprolog{}, you may want to start exploring the distribution tools along different directions.

%=======================================================================
\section{Prolog Programmer Quick Start}
%=======================================================================

As a Prolog programmer, you would like to start trying \tuprolog{} by running your
already existing Prolog programs. You can execute your programs in the form of source
text files using the \tuprolog{} Agent tool. This tool accepts as arguments the name of
a text file containing a Prolog theory and, optionally, the goal to be solved; then it starts
the demonstration. Once you have properly installed \tuprolog{} in the \emph{dir}
directory, you can use the following template to invoke the Agent tool from the
command line:\\\\
%
\texttt{java -cp \emph{dir}/2p.jar\\
\mbox{~~~~~~~~~}alice.tuprolog.Agent \textit{PrologTextFile}
\{\textit{Goal}\}\\\\}
%
For instance, suppose a text file named \verb|hello.pl| in your current directory contains the following line:
\begin{verbatim}
go :- write('hello, world!'), nl.
\end{verbatim}
In order to execute this Prolog program, you can type at the command prompt:\\\\
%
\texttt{java -cp \emph{dir}/2p.jar alice.tuprolog.Agent hello.pl go.\\\\}
%
Then, the Agent tool tries to prove the goal \texttt{go} with respect to the theory contained in \texttt{hello.pl}. 
%
As a result, the string \texttt{hello, world!} should appear on your standard output.

Also, the goal to be proven can be embedded within the Prolog source by means of the \texttt{solve} directive.
%
For instance, suppose that the text file \texttt{hellogo.pl} in your current directory contains the following lines:
\begin{verbatim}
:- solve(go).
go :- write('hello, world!'), nl.
\end{verbatim}
Then, type:\\\\
%
\texttt{java -cp \emph{dir}/2p.jar alice.tuprolog.Agent hellogo.pl\\\\}
%
Again, this will make \texttt{hello, world!} appear on your standard output.

%=======================================================================
\section{Developer Quick Start}
%=======================================================================

The first thing you may want to do as a developer would probably be to take advantage
of the tools embedded in the Graphical User Interface included in the \tuprolog{}
distribution. The GUI can be obtained by issuing the following command:\\\\
%
\texttt{java -cp \emph{dir}/2p.jar alice.tuprologx.ide.GUILauncher\\\\}
%
The development environment provided by the GUI makes standard Prolog features
easily accessible, such as asking queries, viewing the current solution along
with the related variable substitution, backtracking, and so on. Also, it
enables you to view and edit the current Prolog theory contained in the engine,
and to spy \tuprolog{} at work during goal demonstrations. Finally, it also
offers a facility to dynamically load and unload predicate libraries within the
\tuprolog{} engine.

It is worth remembering that the file \texttt{2p.jar} is an executable Java Archive, so
by invoking the command:\\\\
%
\texttt{java -jar 2p.jar\\\\}
%
in the \emph{dir} directory, or by double-clicking it under most operating systems, the
graphic user interface console is automatically spawned.

You may also want to experience a pure interactive environment on a \tuprolog{}
engine. In this case, you need to get the \tuprolog{} prompt using the command line
shell provided within the distribution. To obtain it, just type:\\\\
%
\texttt{java -cp \emph{dir}/2p.jar alice.tuprologx.ide.CUIConsole\\\\}
%
\noindent which starts a \tuprolog{} interpreter to be used via console, in a sort of
Command Line User Interface mode. To exit the \tuprolog{} console, you have to
issue the standard \texttt{halt.} command.
